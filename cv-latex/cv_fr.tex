%%%%%%%%%%%%%%%%%
% This is an sample CV template created using altacv.cls
% (v1.7.2, 28 August 2024) written by LianTze Lim (liantze@gmail.com). Compiles with pdfLaTeX, XeLaTeX and LuaLaTeX.
%
%% It may be distributed and/or modified under the
%% conditions of the LaTeX Project Public License, either version 1.3
%% of this license or (at your option) any later version.
%% The latest version of this license is in
%%    http://www.latex-project.org/lppl.txt
%% and version 1.3 or later is part of all distributions of LaTeX
%% version 2003/12/01 or later.
%%%%%%%%%%%%%%%%

%% Use the "normalphoto" option if you want a normal photo instead of cropped to a circle
% \documentclass[10pt,a4paper,normalphoto]{altacv}

\documentclass[9pt,a4paper,ragged2e,withhyper]{altacv}
%% AltaCV uses the fontawesome5 and simpleicons packages.
%% See http://texdoc.net/pkg/fontawesome5 and http://texdoc.net/pkg/simpleicons for full list of symbols.

% Change the page layout if you need to
\geometry{left=1.25cm,right=1.25cm,top=1.5cm,bottom=1.5cm,columnsep=1.2cm}

% The paracol package lets you typeset columns of text in parallel
\usepackage{paracol}

% Change the font if you want to, depending on whether
% you're using pdflatex or xelatex/lualatex
% WHEN COMPILING WITH XELATEX PLEASE USE
% xelatex -shell-escape -output-driver="xdvipdfmx -z 0" sample.tex
\iftutex 
  % If using xelatex or lualatex:
  \setmainfont{Roboto Slab}
  \setsansfont{Lato}
  \renewcommand{\familydefault}{\sfdefault}
\else
  % If using pdflatex:
  \usepackage[rm]{roboto}
  \usepackage[defaultsans]{lato}
  % \usepackage{sourcesanspro}
  \renewcommand{\familydefault}{\sfdefault}
\fi

% Change the colours if you want to
\definecolor{SlateGrey}{HTML}{2E2E2E}
\definecolor{LightGrey}{HTML}{666666}
\definecolor{DarkPastelRed}{HTML}{450808}
\definecolor{PastelRed}{HTML}{8F0D0D}
\definecolor{GoldenEarth}{HTML}{E7D192}
\colorlet{name}{black}
\colorlet{tagline}{PastelRed}
\colorlet{heading}{DarkPastelRed}
\colorlet{headingrule}{GoldenEarth}
\colorlet{subheading}{PastelRed}
\colorlet{accent}{PastelRed}
\colorlet{emphasis}{SlateGrey}
\colorlet{body}{LightGrey}

% Change some fonts, if necessary
\renewcommand{\namefont}{\Huge\rmfamily\bfseries}
\renewcommand{\personalinfofont}{\footnotesize}
\renewcommand{\cvsectionfont}{\LARGE\rmfamily\bfseries}
\renewcommand{\cvsubsectionfont}{\large\bfseries}


% Change the bullets for itemize and rating marker
% for \cvskill if you want to
\renewcommand{\cvItemMarker}{{\small\textbullet}}
\renewcommand{\cvRatingMarker}{\faCircle}
% ...and the markers for the date/location for \cvevent
% \renewcommand{\cvDateMarker}{\faCalendar*[regular]}
% \renewcommand{\cvLocationMarker}{\faMapMarker*}


% If your CV/résumé is in a language other than English,
% then you probably want to change these so that when you
% copy-paste from the PDF or run pdftotext, the location
% and date marker icons for \cvevent will paste as correct
% translations. For example Spanish:
% \renewcommand{\locationname}{Ubicación}
% \renewcommand{\datename}{Fecha}


%% Use (and optionally edit if necessary) this .tex if you
%% want to use an author-year reference style like APA(6)
%% for your publication list
% \input{pubs-authoryear.tex}

%% Use (and optionally edit if necessary) this .tex if you
%% want an originally numerical reference style like IEEE
%% for your publication list
\input{pubs-num.tex}

%% sample.bib contains your publications
\addbibresource{sample.bib}
% \usepackage{academicons}\let\faOrcid\aiOrcid
\begin{document}
\name{Maxime Pires}
\tagline{Ingénieur IA | CentraleSupelec}
%% You can add multiple photos on the left or right
\photoL{2.8cm}{portrait}

\sentences{
Ingénieur IA diplômé de CentraleSupelec avec une expertise en \textbf{Deep Learning} (Architecture bas niveau \& NLP). Capacité démontrée à construire des solutions ML complexes et à optimiser des pipelines de données. Passionné par l'alliance entre rigueur mathématique et ingénierie logicielle.
}

\personalinfo{%
  % Not all of these are required!
  \email{maximepires4@gmail.com}
  \phone{07 70 44 82 97}
  %\location{Paris / Lyon}
  \linkedin{maximepires}
  \github{maximepires4}
  %% You can add your own arbitrary detail with
  %% \printinfo{symbol}{detail}[optional hyperlink prefix]
  % \printinfo{\faPaw}{Hey ho!}[https://example.com/]

  %% Or you can declare your own field with
  %% \NewInfoFiled{fieldname}{symbol}[optional hyperlink prefix] and use it:
  % \NewInfoField{gitlab}{\faGitlab}[https://gitlab.com/]
  % \gitlab{your_id}
  %%
  %% For services and platforms like Mastodon where there isn't a
  %% straightforward relation between the user ID/nickname and the hyperlink,
  %% you can use \printinfo directly e.g.
  % \printinfo{\faMastodon}{@username@instace}[https://instance.url/@username]
  %% But if you absolutely want to create new dedicated info fields for
  %% such platforms, then use \NewInfoField* with a star:
  % \NewInfoField*{mastodon}{\faMastodon}
  %% then you can use \mastodon, with TWO arguments where the 2nd argument is
  %% the full hyperlink.
  % \mastodon{@username@instance}{https://instance.url/@username}
}

\makecvheader
%% Depending on your tastes, you may want to make fonts of itemize environments slightly smaller
% \AtBeginEnvironment{itemize}{\small}

%% Set the left/right column width ratio to 6:4.
\columnratio{0.7}

% Start a 2-column paracol. Both the left and right columns will automatically
% break across pages if things get too long.
\begin{paracol}{2}
\cvsection{Expériences}

\cvevent{Ingénieur IA - Stage}{Ministère des Armées}{Septembre 2024 -- Février 2025}{Paris}
\textbf{R\&D sur l'analyse sémantique et syntaxique de corpus via Word Embeddings.}
\linebreak
\begin{itemize}
\item Conception d'un pipeline NLP end-to-end pour l'analyse sémantique de corpus massif (plusieurs Go) : du pré-traitement (nettoyage, tokenisation) à la vectorisation optimisée.
\item Analyse comparative de l'état de l'art (SOTA) sur les modèles de langage (Word2Vec, BERT, GPT) pour orienter les choix architecturaux vers la solution la plus robuste.
\item Optimisation de la représentation vectorielle : Développement de métriques d'évaluation sur-mesure (erank, edim) pour maximiser la densité informationnelle et réduire la dimensionnalité des vecteurs.
\item \textit{Stack : Python, Scikit-learn, Gensim, Transformers, Pandas.}
\end{itemize}

\divider

\cvevent{Ingénieur Data \& Full Stack - Stage}{Ministère des Armées}{Février 2023 -- Août 2023}{Paris}
\textbf{Création d'un outil pour faciliter le traitement de données scientifiques.}
\linebreak
\begin{itemize}
\item Développement d'une application web scientifique (Next.js) pour centraliser et fluidifier l'analyse de données critiques.
\item Optimisation drastique de la performance : Mise en place d'algorithmes permettant le traitement automatisé de milliers de dossiers en quelques secondes, remplaçant une analyse manuelle coûteuse.
\item Conception d'une base de données relationnelle garantissant l'intégrité des données scientifiques et un accès temps réel aux résultats (PostgreSQL).
\end{itemize}

\medskip

\cvsection{Projets}

\cvevent{\faBrain[regular] MPNeuralNetwork}{Librairie Deep Learning from scratch en Python/NumPy vectorisée}{}{}
\begin{itemize}
  \item Implémentation manuelle des Layers (1D \& 2D avec `im2col`), Optimizers (Backpropagation), Losses, Activations, et Metrics.
  \item Moteur "intelligent" pour automatiser les meilleures pratiques du Deep Learning (early stopping, checkpoint, initialisation des poids, régularisation)
  \item Optimisation des performances : x4 speedup grâce à la vectorisation complète (Batch processing).
  \item Gestion de la stabilité numérique (Log-Sum-Exp trick).
  \item \textbf{Industrialisation \& Qualité} : Pipeline CI/CD automatisé (GitHub Actions), typage statique (MyPy), linting moderne (Ruff, Pre-commit) et tests unitaires (Pytest).
\end{itemize}

\divider

\cvevent{\faEye[regular] Handwriting Recognition}{Application GUI temps réel de reconnaissance de chiffres (MNIST)}{}{}
\begin{itemize}
    \item Application direct de MPNeuralNetwork.
    \item Précision de 98\% avec un réseau Fully Connected, 99\% avec un CNN.
    \item Pipeline de pré-traitement d'image (Centrage de masse, Redimensionnement) similaire aux standards MNIST.
\end{itemize}

%% Switch to the right column. This will now automatically move to the second
%% page if the content is too long.
\switchcolumn

\medskip

\cvsection{Formations}

\cvevent{CentraleSupelec}{Intelligence Artificielle}{2023 -- 2025}{Paris}
Double diplôme

\divider

\cvevent{ECE - École d'ingénieur}{Sciences de l'information}{2019 -- 2023}{Paris}
Section internationale

\medskip

\cvsection{Compétences}

% Don't overuse these \cvtag boxes — they're just eye-candies and not essential. If something doesn't fit on a single line, it probably works better as part of an itemized list (probably inlined itemized list), or just as a comma-separated list of strengths.

% The `ragged2e` document class option might cause automatic linebreaks between \cvtag to fail.
% Either remove the ragged2e option; or 
% add \LaTeXraggedright in the paragraph for these \cvtag

\cvevent{IA \& Data Science}{
\cvtag{Python}
\cvtag{Scikit-learn}
\cvtag{SQL}
\cvtag{PyTorch}
\cvtag{Pandas}
\cvtag{NumPy}
}{}{}

\divider

\cvevent{Engineering \& DevOps}{
\cvtag{Git}
\cvtag{Linux}
\cvtag{Docker}
}{}{}

\divider

\cvevent{Web \& App}{
\cvtag{React}
\cvtag{Next.js}
\cvtag{TypeScript}
\cvtag{C++}
\cvtag{Java}
}{}{}

\medskip

\cvsection{Langues}

\cvachievement{C2}{Français}{Langue maternelle}

\divider

\cvachievement{C1}{Anglais}{TOEIC 965/990}

%% Yeah I didn't spend too much time making all the
%% spacing consistent... sorry. Use \smallskip, \medskip,
%% \bigskip, \vspace etc to make adjustments.
\medskip

\cvsection{Intérêts}

\cvachievement{$\bullet$}{Informatique}{Open-source, Unix}

\divider

\cvachievement{$\bullet$}{Sport}{Handball, Vélo, Plongée}

\divider

\cvachievement{$\bullet$}{Vulgarisation scientifique}{Maths, Physique, Biologie}

\divider

\cvachievement{$\bullet$}{Art}{Cinéma, Lecture, Musique}

\end{paracol}

\end{document}

%%%%%%%%%%%%%%%%%
% This is an sample CV template created using altacv.cls
% (v1.7.2, 28 August 2024) written by LianTze Lim (liantze@gmail.com). Compiles with pdfLaTeX, XeLaTeX and LuaLaTeX.
%
%% It may be distributed and/or modified under the
%% conditions of the LaTeX Project Public License, either version 1.3
%% of this license or (at your option) any later version.
%% The latest version of this license is in
%%    http://www.latex-project.org/lppl.txt
%% and version 1.3 or later is part of all distributions of LaTeX
%% version 2003/12/01 or later.
%%%%%%%%%%%%%%%%

%% Use the "normalphoto" option if you want a normal photo instead of cropped to a circle
% \documentclass[10pt,a4paper,normalphoto]{altacv}

\documentclass[9pt,a4paper,ragged2e,withhyper]{altacv}
%% AltaCV uses the fontawesome5 and simpleicons packages.
%% See http://texdoc.net/pkg/fontawesome5 and http://texdoc.net/pkg/simpleicons for full list of symbols.

% Change the page layout if you need to
\geometry{left=1.25cm,right=1.25cm,top=1.5cm,bottom=1.5cm,columnsep=1.2cm}

% The paracol package lets you typeset columns of text in parallel
\usepackage{paracol}

% Change the font if you want to, depending on whether
% you're using pdflatex or xelatex/lualatex
% WHEN COMPILING WITH XELATEX PLEASE USE
% xelatex -shell-escape -output-driver="xdvipdfmx -z 0" sample.tex
\iftutex 
  % If using xelatex or lualatex:
  \setmainfont{Roboto Slab}
  \setsansfont{Lato}
  \renewcommand{\familydefault}{\sfdefault}
\else
  % If using pdflatex:
  \usepackage[rm]{roboto}
  \usepackage[defaultsans]{lato}
  % \usepackage{sourcesanspro}
  \renewcommand{\familydefault}{\sfdefault}
\fi

% Change the colours if you want to
\definecolor{SlateGrey}{HTML}{2E2E2E}
\definecolor{LightGrey}{HTML}{666666}
\definecolor{DarkPastelRed}{HTML}{450808}
\definecolor{PastelRed}{HTML}{8F0D0D}
\definecolor{GoldenEarth}{HTML}{E7D192}
\colorlet{name}{black}
\colorlet{tagline}{PastelRed}
\colorlet{heading}{DarkPastelRed}
\colorlet{headingrule}{GoldenEarth}
\colorlet{subheading}{PastelRed}
\colorlet{accent}{PastelRed}
\colorlet{emphasis}{SlateGrey}
\colorlet{body}{LightGrey}

% Change some fonts, if necessary
\renewcommand{\namefont}{\Huge\rmfamily\bfseries}
\renewcommand{\personalinfofont}{\footnotesize}
\renewcommand{\cvsectionfont}{\LARGE\rmfamily\bfseries}
\renewcommand{\cvsubsectionfont}{\large\bfseries}


% Change the bullets for itemize and rating marker
% for \cvskill if you want to
\renewcommand{\cvItemMarker}{{\small\textbullet}}
\renewcommand{\cvRatingMarker}{\faCircle}
% ...and the markers for the date/location for \cvevent
% \renewcommand{\cvDateMarker}{\faCalendar*[regular]}
% \renewcommand{\cvLocationMarker}{\faMapMarker*}


% If your CV/résumé is in a language other than English,
% then you probably want to change these so that when you
% copy-paste from the PDF or run pdftotext, the location
% and date marker icons for \cvevent will paste as correct
% translations. For example Spanish:
% \renewcommand{\locationname}{Ubicación}
% \renewcommand{\datename}{Fecha}


%% Use (and optionally edit if necessary) this .tex if you
%% want to use an author-year reference style like APA(6)
%% for your publication list
% \input{pubs-authoryear.tex}

%% Use (and optionally edit if necessary) this .tex if you
%% want an originally numerical reference style like IEEE
%% for your publication list
\input{pubs-num.tex}

%% sample.bib contains your publications
\addbibresource{sample.bib}
% \usepackage{academicons}\let\faOrcid\aiOrcid
\begin{document}
\name{Maxime Pires}
\tagline{AI Engineer | CentraleSupelec}
%% You can add multiple photos on the left or right
\photoL{2.8cm}{portrait}

\sentences{
AI Engineer graduated from CentraleSupelec with expertise in \textbf{Deep Learning} (Low-level Architecture \& NLP). Proven ability to build complex ML solutions and optimize data pipelines. Passionate about combining mathematical rigor with software engineering.
}

\personalinfo{% 
  % Not all of these are required!
  \email{maximepires4@gmail.com}
  \phone{+33 7 70 44 82 97}
  %
  \linkedin{maximepires}
  \github{maximepires4}
  %% You can add your own arbitrary detail with
  %% \printinfo{symbol}{detail}[optional hyperlink prefix]
  % \printinfo{\faPaw}{Hey ho!}[https://example.com/]

  %% Or you can declare your own field with
  %% \NewInfoFiled{fieldname}{symbol}[optional hyperlink prefix] and use it:
  % \NewInfoField{gitlab}{\faGitlab}[https://gitlab.com/]
  % \gitlab{your_id}
  %%
  %% For services and platforms like Mastodon where there isn't a
  %% straightforward relation between the user ID/nickname and the hyperlink,
  %% you can use \printinfo directly e.g.
  % \printinfo{\faMastodon}{@username@instace}[https://instance.url/@username]
  %%
  %% But if you absolutely want to create new dedicated info fields for
  %% such platforms, then use \NewInfoField* with a star:
  % \NewInfoField*{mastodon}{\faMastodon}
  %%
  %% then you can use \mastodon, with TWO arguments where the 2nd argument is
  %% the full hyperlink.
  % \mastodon{@username@instance}{https://instance.url/@username}
}

\makecvheader
%% Depending on your tastes, you may want to make fonts of itemize environments slightly smaller
% \AtBeginEnvironment{itemize}{\small}

%% Set the left/right column width ratio to 6:4.
\columnratio{0.7}

% Start a 2-column paracol. Both the left and right columns will automatically
% break across pages if things get too long.
\begin{paracol}{2}
\cvsection{Experience}

\cvevent{AI Engineer - Intern}{French Ministry of Armed Forces}{September 2024 -- February 2025}{Paris}
\textbf{R\&D on semantic and syntactic corpus analysis via Word Embeddings.}
\linebreak
\begin{itemize}
\item Design of an end-to-end NLP pipeline for massive corpus semantic analysis (several GB): from pre-processing (cleaning, tokenization) to optimized vectorization.
\item Comparative analysis of state-of-the-art (SOTA) language models (Word2Vec, BERT, GPT) to guide architectural choices towards the most robust solution.
\item Vector representation optimization: Development of custom evaluation metrics (erank, edim) to maximize informational density and reduce vector dimensionality.
\item \textit{Stack: Python, Scikit-learn, Gensim, Transformers, Pandas.}
\end{itemize}

\divider

\cvevent{Data \& Full Stack Engineer - Intern}{French Ministry of Armed Forces}{February 2023 -- August 2023}{Paris}
\textbf{Creation of a tool to facilitate scientific data processing.}
\linebreak
\begin{itemize}
\item Development of a scientific web application (Next.js) to centralize and streamline critical data analysis.
\item Drastic performance optimization: Implementation of algorithms allowing automated processing of thousands of folders in seconds, replacing costly manual analysis.
\item Design of a relational database ensuring scientific data integrity and real-time access to results (PostgreSQL).
\end{itemize}

\medskip

\cvsection{Projects}

\cvevent{\faBrain[regular] MPNeuralNetwork}{Deep Learning library built from scratch in Python/NumPy (vectorized)}{}{}
\begin{itemize}
    \item Manual implementation of Layers (1D \& 2D with `im2col`), Optimizers (Backpropagation), Losses, Activations, and Metrics.
    \item ``Smart'' engine to automate Deep Learning best practices (early stopping, checkpoints, weight initialization, regularization).
    \item Performance optimization: x4 speedup thanks to full vectorization (Batch processing).
    \item Numerical stability management (Log-Sum-Exp trick).
    \item \textbf{Industrialization \& Quality}: Automated CI/CD pipeline (GitHub Actions), static typing (MyPy), modern linting (Ruff, Pre-commit), and unit tests (Pytest).
\end{itemize}

\divider

\cvevent{\faEye[regular] Handwriting Recognition}{Real-time GUI application for digit recognition (MNIST)}{}{}
\begin{itemize}
    \item Direct application of MPNeuralNetwork.
    \item 98\% accuracy with a Fully Connected network, 99\% with a CNN.
    \item Image pre-processing pipeline (Center of mass, Resizing) similar to MNIST standards.
\end{itemize}

%% Switch to the right column. This will now automatically move to the second
%% page if the content is too long.
\switchcolumn

\medskip

\cvsection{Education}

\cvevent{CentraleSupelec}{Artificial Intelligence}{2023 -- 2025}{Paris}
Double Degree

\divider

\cvevent{ECE - Engineering School}{Information Systems}{2019 -- 2023}{Paris}
International Section

\medskip

\cvsection{Skills}

% Don't overuse these \cvtag boxes — they're just eye-candies and not essential. If something doesn't fit on a single line, it probably works better as part of an itemized list (probably inlined itemized list), or just as a comma-separated list of strengths.

% The `ragged2e` document class option might cause automatic linebreaks between \cvtag to fail.
% Either remove the ragged2e option; or 
% add \LaTeXraggedright in the paragraph for these \cvtag

\cvevent{AI \& Data Science}{
\cvtag{Python}
\cvtag{Scikit-learn}
\cvtag{SQL}
\cvtag{PyTorch}
\cvtag{Pandas}
\cvtag{NumPy}
}{}

\divider

\cvevent{Engineering \& DevOps}{
\cvtag{Git}
\cvtag{Linux}
\cvtag{Docker}
}{}

\divider

\cvevent{Web \& App}{
\cvtag{React}
\cvtag{Next.js}
\cvtag{TypeScript}
\cvtag{C++}
\cvtag{Java}
}{}

\medskip

\cvsection{Languages}

\cvachievement{C2}{French}{Native}

\divider

\cvachievement{C1}{English}{TOEIC 965/990}

%% Yeah I didn't spend too much time making all the
%% spacing consistent... sorry. Use \smallskip, \medskip,
%% \bigskip, \vspace etc to make adjustments.
\medskip

\cvsection{Interests}

\cvachievement{\textbullet}{Computing}{Open-source, Unix}

\divider

\cvachievement{\textbullet}{Sports}{Handball, Cycling, Diving}

\divider

\cvachievement{\textbullet}{Scientific Outreach}{Maths, Physics, Biology}

\divider

\cvachievement{\textbullet}{Arts}{Cinema, Reading, Music}

\end{paracol}

\end{document}
